\documentclass[12pt]{article}

% fonts

\usepackage[T1]{fontenc}
\usepackage[full]{textcomp}
\usepackage{newtxtext}
\usepackage{cabin} % sans serif
\usepackage[varqu,varl]{inconsolata} % sans serif typewriter
\usepackage[final,expansion=alltext]{microtype}
\usepackage[english]{babel}
\usepackage{amsmath}
\usepackage[bigdelims,vvarbb]{newtxmath} % bb from STIX
\usepackage[cal=boondoxo]{mathalfa} % mathcal
\usepackage{csquotes}

\usepackage[backend=bibtex,style=numeric]{biblatex}
\bibliography{traffic} % Specify your BibTeX file

% geometry of the page

\usepackage[top=1in,
            bottom=1in,
            left=1in,
            right=1in]{geometry}


% paragraph spacing

\setlength{\parindent}{0pt}
\setlength{\parskip}{2ex plus 0.4ex minus 0.2ex}


\author{Matthew McAnear}
\title{Traffic Fatality Risk by Advanced Safety Features}

% useful packages

\begin{document}

\maketitle

\begin{abstract}
    Despite the proliferation of AI technology in passenger cars of collision detection/avoidance, lane drift systems, and
    various other safety systems, pedestrian and cyclist deaths are NOT significantly reduced by these
    interventions. 
\end{abstract}


\section{Introduction}

Traffic saftey is a major concern for all roadway users. The National Highway Traffic Safety Administration (NHTSA) reports
that, as of preliminary reports for through June of 2023, traffic accidents involving fatalities, in aggregate,
are down about 3\%. This decrease hides the true scale of the problem for non-road users, however. While the
overall change for pedestrians is also 3\%, the baseline level of risk for pedestrians is much higher than for 
passenger vehicles or even for pedalcyclists. Accidents involving pedestrians range anywhere from a 14\% to 
23\% baseline fatality rate, depending on the month in question \cite{national_center_for_statistics_and_analysis_early_2024}.

Among drivers in the vehicle, these fatality rates have been decreasing over time, with cars becoming
safer for occupants of the vehicle. However, the overall fatality is influenced heavily by the mass of the 
car\cite{evans_car_1992}. This means that there are offsetting effects of both increasing vehicle mass and increasing 
vehicle saftey. For people outside of the vehicle, however, the statistics are alarming. According to Tyndall, 
"between 2010 and 2021 the number of pedestrians killed annually in collisions increased by 72\%, from 4300 to 
7400 \cite{tyndall_effect_2024}." Due to relaxed emissions standards for small trucks compared to passenger cars,
American buyer preferences shifted toward sport utility vehicles (SUVs) and trucks \cite{kovach_rise_2021}. Taken into 
account this connection between both the proliferation of large vehicles and the clear connection in fatality risk to
more massive and taller vehicles\cite{tyndall_effect_2024}, finding the factors that minimize pedestrian fatalities
is of critical importance for public health.

It is common knowledge that AI systems have become an increasing part of every day life, and passenger
vehicles are no exception. The question is if these improved safety systems are reducing the risk of fatalities
among pedestrians and cyclists. Next, assuming these new systems yield tangible improvements to pedestrian safety,
are the improvements large enough in magnitude to offset the increased risk from the size and height of vehicles.

In this paper, we focus primarily on those factors that influence the risk of fatalities in crashes involving pedestrians
and cyclists. We exclude from consideration crashes involving multiple vehicles and focus solely on single-vehicle 
collisions with pedestrians and cyclists. After controlling for the effect of weather conditions,
geographic region, and crash variables such as time of day and intersection type, we estimate the effect
of advanged safety features on the risk of pedestrian and cyclist fatalities within vehicle types.


\section{Data}

Our data comes the 

\section{Methodology}

\section{Results}


\section{Discussion}

\section{Conclusion}

\printbibliography

\end{document}

%%% Local Variables:
%%% mode: latex
%%% TeX-master: t
%%% End:
